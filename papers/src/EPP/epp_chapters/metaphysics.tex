\section{The Metaphysics of the Effect Propagation Process}
\label{sec:metaphysics}

The metaphysics of the Effect Propagation Process establishes a set of underlying principles widely used across the EPP. The three core metaphysical concepts are defined as: Monoidic Primitives, Isomorphic Recursive Composition, and Contextual Relativity. Then we derive from these first principles the metaphysics of being, dynamics, becoming, and doing. Combined, these form the foundation of orthogonal design used throughout the EPP and its implementation DeepCausality. Ontological design is a constructive, engineering-oriented form of philosophy to specify the necessary and sufficient conditions of a new system to exist and operate coherently. The EPP's metaphysics is, therefore, the blueprint for a computable reality. It establishes the axiomatic foundation upon which the ontology is built, the epistemology is derived, and the implementation is realized.

\subsection{Monoidic Primitives} 
\label{sec:metaphysics_monoidic_primitives}

A monoid is defined as an abstract algebraic structure that comprises:

\begin{enumerate}
	\item A set of elements with a certain type.
	\item A binary operation that combines any two elements of the set results in a third element of the same set.
	\item An identity element, which, when combined with any other element, leaves it unchanged.
\end{enumerate}

  Monoidic elements that can be combined with each other are fundamental to the composability of the EPP.
  
\subsection{Isomorphic Recursive Composition} 
\label{sec:metaphysics_isomorphic_recursive_composition}


Isomorphism means "having the same shape” in the sense that isomorphic elements all share the same form. Recursion refers to a structure containing itself. Isomorphism enables the combination of different types of monoidic primitives whereas recursion allows self-referential nesting. Combining these two concepts results in isomorphic recursive composition. This composition enables a monoidic primitive to derive its significance from its structural relation to other monoidic primitives. Critically, to ensure the non-reducibility of form of monoidic primitives within the isomorphic recursive composition, it must have a singleton representation of itself. 

\subsection{The Metaphysics of Context} 
\label{sec:metaphysics_context}

The metaphysics of context is structured as a classical "matter", the hyle, through the
monoidic primitive of the contextoid. The contextoid entails one unit of context, that can be
space, time, spacetime, symbol, or data. The context is free of recursion to prevent paradoxical states such as time-loops. Furthermore, the metaphysics does not prescribe any particular shape or structure of the context to ensure a context can be represented in a structure best suited for the domains at hand.


\subsection{The Metaphysics of Being} 
\label{sec:metaphysics_being}

The metaphysics of being is structured in a classical Aristotelian notion (Book Zeta\cite{furth1985metaphysics}): 

\begin{itemize}
	\item Monoidic Primitives (Matter): The EPP is composed of fundamental, identifiable elements.
	\item Isomorphic Recursive Composition (Form): Gives the primitive elements its form. 
\end{itemize}

Combined, the monoidic primitives and the isomorphic recursive composition form an Aristotelian hylomorphic compound. The `Monoidic Primitives` constitute the "matter" (hyle) and the `Isomorphic Recursive Composition` provides the "form" (morphe) that arranges its elements into a structured, meaningful whole.

This hylomorphic compound is the  \textit{archê kai aitia} of the system's existence. The specific form imposed upon the primitives is the foundational principle and explanation for its properties. Therefore, the hylomorphic compound is by definition static and describes what the system is at a snapshot in time, but it contains no inherent mechanism of change. 

\newpage

\subsection{The Metaphysics of Dynamics} 
\label{sec:metaphysics_dynamics}

The classical hylomorphism describes a static substance as the combination of its matter and form. Therefore, to account for the dynamism inherent in the EPP, the \textit{archê kai aitia} needs to capture the dynamics within an existing substance. However, because of the spacetime agnostic design of the EPP, dynamics can only be defined relative to its engulfing context. Therefore, the principle of contextual relativity operationalizes the expression of Substantial-Structural co-determination relative to its context. This is an inherent  principle of how a substance with a fixed identity can exhibit variable states. By its definition, dynamics is predictable and bound to an existing substance and context. 

The dynamics in the EPP operates relative to its engulfing context and within its structure. The structure might be static or dynamic, but for dynamic the separation between hyle (context), logos (logic) and its structure remains intact. That means, a causal logic can only operate along pathways. Dynamics cannot leave a pathway, it must follow it regardless of whether these paths were established upfront or created dynamically

\subsection{The Metaphysics of Adaptation} 
\label{sec:metaphysics_adaptation}

 For Adaptive dynamics as in adaptive reasoning, there is an element of interaction between the logos and its structure. The logos, a causaloid, can alter adaptively the pathway through the causal structure depending on its internal reasoning. Specifically, when a causal model is structured in such a way that it contains sub-models for specific scenarios, then a causaloid in adaptive reasoning mode can determine dynamically which sub-model is more applicable and then dispatch the reasoning path to the sub-model and with that alter the causal reasoning pathway. Notice, as long as the targets of the dispatch exist and the dispatch logic is decidable, the modality of adaptive dynamics remains deterministic. Although adaptive dynamics can change the reasoning pathway, however, it cannot fundamentally change or evolve the structure itself.  


\subsection{The Metaphysics of Structural Change} 
\label{sec:metaphysics_becoming}

The principle of contextual relativity (Dynamics) accounts for how a system can change its state predictably within a fixed structure and enables reasoning along a fixed path. Adaptive reasoning extends the  modality towards a dynamic pathway through a static or dynamic structure. However, none of it can account for how the structure itself might evolve, as is required in advanced use cases like autonomous navigation. For structural change, the principle of emergence allows for two modalities.

\begin{itemize}
	\item Fixed structural change
	\item Dynamic structural change
\end{itemize}

Fixed structural changes limit systems to pre-defined all possible structural changes that are triggered by known conditions.
Consequently, the system remains fully deterministic and verifiable, but at the expense of adaptability. 
Conversely, dynamic structural changes enable adaptability by  generating novel causal structures that were not explicitly encoded before,
but at the loss of determinism and verifiability. This fundamental trade-off determinism at the expense of adaptability versus
adaptability at the expense of  determinism demands a fundamental decision. Also, fixed structural change can be expressed 
as a specialized form of dynamic structural change with a momentum of change held constant. The reverse, however, is not possible. 

The Effect Propagation Process as a theory of dynamic causality deliberately embraces dynamic structural changes.
The limitations of fixed structural changes are too restrictive for the next generation of intelligent systems.
The consequences of defining and implementing dynamic structural changes, while substantial, 
are accepted as the price of enabling truly dynamic causality in a dynamic environment. 
Therefore, the metaphysics of becoming defines the required principle of emergence, 
the subsequent ontology structures the mechanism of emergence, and the epistemology captures 
the implications. To capture dynamic structural change, the principle of Higher-Order Emergence  becomes necessary.

\subsection{The Metaphysics of Becoming} 
\label{sec:metaphysics_becoming}

The principle of Higher-Order Emergence captures the profound process of creation of new substance from within an existing substance. 
Emergence brings into being a new matter, a new form, both of them, or new dynamics. The principle of Higher-Order Emergence operationalizes two different modalities.  

First-Order Emergence explains the creation of new substance. While contextual relativity applies to an existing substance, it cannot bring into being a new substance. The principle of `First-Order Emergence` posits a generative capacity within the EPP, a capability of imposing a novel \textit{archê kai aitia} upon a set of primitives. This is the mechanism by which a new substance from within an existing context comes into being.

In this modality, the distinction between the system and its context begins to dissolve. The result is a non-deterministic co-evolution where the spectrum of subsequent causal structures and contextual facts cannot be predicted any longer. The "reason for being" \textit{aitia} of any given state is no longer a fixed principle but is itself an emergent property of the ongoing, self-modifying process. However, the process of becoming is invariant because of the first-order designation. 

Higher-Order Emergence moves beyond the creation of a new substance from within an existing one. Instead, it describes a state where the generative capacity that enables `Emergence` acts upon itself recursively in what amounts to a dynamic co-mergence of form, matter, and dynamics through recursive higher order emergence. 

In this modality, the process of becoming itself becomes dynamic. This higher order emergence represents the EPP's most advanced state of becoming, one that necessitates a new epistemology of emergence to explore its inherent emergent properties.


Metaphysically, the principle of Higher-Order Emergence demands further elaboration. Fundamentally, it implies that first-order emergence, the capacity to create new substance, is simply a less recursive specialization of the governing higher order principle.The principle of Higher-Order Emergence leads to outcomes that are not necessarily guaranteed to be decidable let alone deterministic any longer and therefore it foreshadows three crises:

\begin{enumerate}
	\item The Crisis of Justification
	\item The Crisis of Truth
	\item The Crisis of Explainability
\end{enumerate}


The crisis of justification immediately results from the fact that it must be decided how to choose one state of emergence over another and how. The justification must be there otherwise the decision cannot be made, but this would render higher-order emergence fundamentally undecidable. 

The crisis of truth results from the fact that, if emergence generates a new context, then how do we know that
the facts in the newly generated context are true? If emergence generates new causal rules that uses new facts from a generated context, how do we know the outcome is true? If facts are fluid, verification is impossible. Therefore, truth must be re-established otherwise it undermines trust in operational safety of the EPP. 

The crisis of explainability means that in co-emergence, it might not be possible any longer to explain the outcome because of the previous crises of truth and the crisis of justification. Furthermore, if the process of emergence itself cannot be explained, how could possible derived artifacts be explained? How can a system be held accountable when it's not explainable. It is not possible, and therefore the crisis of explainability roots in the very core of higher-order emergence.

The introduction of higher order emergence also raises the question of the genesis process, 
the origin of the emergence itself. Fundamentally, the genesis process imposes a decision: Do we allow any kind of machine intelligence to modify its genesis process or not? The author argues for an unequivocal no. 
Considering the alternative, when a system that can evolve its own genesis process, it would fundamentally become uncontrollable and unexplainable. 

Therefore, the genesis process of emergence  has to remain at the sole discretion of a human designer to ensure its explainability and a fundamental alignment with human values. The metaphysics itself cannot establish the core ethos or telos, only a human designer can do that. Under no circumstances should the genesis process in parts or in its entirety ever be created or modified by a machine intelligence because it cannot possibly have the innate ethos of a human being and thus cannot possibly align itself with humanity. 

The existence of the genesis process also raises the thorny issue of whose ethics and values to codify and why? Which human designer? Who guards the guardians of the genesis? How to balance conflicts between different stakeholders? These are  immense normative and political challenges and, a metaphysics alone, cannot possibly answer these fundamentally societal questions. 

As a consequence, the genesis process itself is a decision with far reaching higher order effects. The mitigation of inevitable unintended higher order consequences, will lead to the necessity of an immutable genesis telos, an underlying intent that serves as a criterion to discern whether emerging states are intended. 


\subsection{The Metaphysics of Axiology} 
\label{sec:metaphysics_axiology}

The reasoning (logos) alone does not have any impact without action, but action without reason (logos) may not lead to a desired outcome. The link between reasoning and action is captured in axiology, the study of value, ethics, intent, and aesthetics. For the EPP, two aspects of axiology are used. 

The Telos, intent, which is the specific, singular intent or overarching goal applied proactively and the ethos, a framework of rules retrospectively applied. The goal of the telos is to prevent incorrect causal reasoning within a causaloid, for example, resulting from invalid context reading i.e. because of a faulty sensor. 

The ethos with its finer set of rules is applied after all reasoning has been completed, but before an action can be taken. A well known property of complex systems is that, even when all components operate within a valid range, the final outcome can potentially still be invalid because of unforeseen state combinations. Therefore, the goal of the ethos is to prevent unforeseen consequences from happening by vetoing actions that would violate the rules codified in it. 

In the EPP, the causaloid folds cause and effect into one single monoidic entity and the PropagatingEffect folds input and output into one single monoidic entity. In the same way, the teloid folds intent and ethos into one single monoidic entity with a twofold function:

\textbf{Teloid:} A single teloid codifies a single intent or a single rule. A causaloid can query one or more teloids to inform whether its reasoning stays within codified intent to ensure alignment. 

\textbf{Effect Ethos:} Multiple teloids, using isomorphic recursion, form a scalable rule set that codify the effect ethos governing the final PropagatingEffect. Before a reasoning outcome stages into an action, it must pass the governing effect ethos to ensure that the proposed action is within the overall applicable rule set. Only after the checks enforced by the effect ethos have passed, a proposed action can be executed. 

Together, the teloid and the effect ethos safeguard operations of the EPP regardless of the modality of dynamics. Furthermore, the three crises caused by emergence are mitigated:

\begin{enumerate}
	\item The Effect Ethos justifies why an action was taken. 
	\item The Teloid codifies the intent and thus attests to the ground truth.
	\item Both, the Teloid and Effect Ethos explain a system even when it is emergent.
\end{enumerate}

Because dynamics and emergence are first class principles in the EPP, so are the
teloid and effect ethos to safely govern the EPP and its actions. 

\subsection{The Metaphysics of Doing} 
\label{sec:metaphysics_doing}

In the EPP, the transition from insight to action is governed by a causal state machine that is based on two principles.

 First, the causal state that takes as input any terminal PropagatingEffect and decides deterministically whether to act. The separation between reasoning outcome and deciding whether to act upon the reasoning outcome allows for arbitrary complex decision logic especially if the terminal PropagatingEffect is, for example, symbolic or probabilistic. In this causal state, the effect ethos intercepts the line of reasoning and decides whether the final determination to act falls within its rules. 
 
 Second, the causal action encodes what action to take conditional on the causal state. The causal action is an arbitrary and unconstrained function to allow for the broadest possible set of action an EPP system can take. 
 
 Combined, the causal state and action form the causal state machine governed by the effect ethos to safely transition from insight to action while respecting all rules encoded in the accompanying ethos. 

\subsection{Discussion} 
\label{sec:metaphysics_summary}

The complete metaphysics of the EPP is the synthesis of three core principles:

\begin{enumerate}
	\item Context: The hyle, the eternalized context. 
	\item Substance (Being): The hylomorphic compound of Primitives and Composition.
	\item Dynamics (Changing): Governed by the principle of Contextual Relativity.
	\item Adaptive (Adapting): Governed by the principle of adaptive reasoning. 
	\item Emergence (Becoming): Governed by the principle of Higher-Order Emergence. 
	\item Effect Ethos (Judging): Governed by its axiology. 
	\item Acting (Doing): Governed by the effect ethos. 
\end{enumerate}


The EPP's core metaphysical principles, Monoidic Primitives, Isomorphic Recursive Composition combined with the principles of and Contextual Relativity and  Higher-Order Emergence form the foundation of the EPP. The EPP provides a framework that distinguishes between two tiers of emergence. First-Order Emergence describes the system's capacity to generate new  substance from a stable set of generative rules. Higher-Order Emergence, describes the system's ultimate capacity to evolve itself via a recursive and open-ended process of generative emergence. The introduction of higher-order emergence implies that the ultimate outcome of the emergent process is not guaranteed to be decidable let alone deterministic any longer. This is a deliberate design decision to provide multiple modalities for different requirements. For dynamic systems, contextual relativistic dynamics should suffice. For handling regime change where the new structure can be decided a-priori, first-order emergence should suffice. However, when handling dynamic relativistic regime change in response to an evolving context, higher-order emergence becomes necessary to capture the dynamic co-emergence, but it removes the deterministic foundation of causality, which introduces three foundational crises. The introduction of the teloid and effect ethos address these crises by establishing a priori and a posteriori policy check to ensure safe, reliable, and explainable operations. The EPP, through the introduction of Higher-Order Emergence and the effect ethos, establishes a foundation for exploring the interrelation between emergent systems and human value. 

\newpage
