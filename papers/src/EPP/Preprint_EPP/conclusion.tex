\section{Conclusion}
\label{sec:conclusion}

The Effect Propagation Process framework rethinks causality as a continuous transfer of effects originating from a potentially non-spatiotemporal underlying structure by leveraging a single axiomatic generalized definition of causality.

The metaphysics of Effect Propagation Process establishes the essence of the EPP and establishes its core dynamics that lead to an orthogonal design that consistently applies to all levels of the EPP. 

The ontology of the Effect Propagation Process foreshadows the complex structures the EPP is designed to model by scaling the modality of the EPP. For a static EPP, a positivist epistemology remains sufficient. For a dynamic EPP, the epistemology evolves towards an interpretivism perspective, and for an emergent EPP, a pragmatism perspective on the epistemology becomes necessary.

The epistemology of the Effect Propagation Process reflects the complex systems it is designed to model by scaling with the modality of the EPP. For a static EPP, a positivist epistemology remains sufficient. For a dynamic EPP, the epistemology evolves towards an interpretivism perspective, and for an emergent EPP, a pragmatism perspective on the epistemology becomes necessary. Likewise, for the justification of knowledge in an EPP, the underlying notion of truth scales with the modality of the EPP. For a static EPP, the meaning of truth aligns with the classical correspondence theory. However, in a dynamic EPP, the meaning of truth shifts towards a coherent adaptability approach. In an emergent EPP, the meaning of truth evolves towards pragmatic efficacy where the validity of relativistic, emergent causal relationships is established by their functional utility.

The EPP provides a robust foundation for exploring the nature of effect propagation in a universe that may fundamentally defy classical intuition while leaving traditional teleology as a likely emergent property. The Effect Propagation Process offers a unified philosophical language of causality that is powerful enough to handle new challenges, remains compatible with classical causality, and conceptually aligns with contemporary theories of quantum gravity. Future work may explores the realm of dynamic emergent causality further. 


\newpage